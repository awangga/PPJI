\chapter{Hasil Review}

Tak ada gading yang tak retak. Panduan ini juga masih banyak kekurangan, dan perlu banyak jam terbang untuk evaluasi. Pada bagian ini contoh bagaimana sebuah paper mengalami review dan penolakan dari paper yang dikirim kepada jurnal Q1, Q2 dan Q3. 

\section{Artikel Ditolak Q1}
Beberapa contoh artikel ditolak dengan isi komentar :
\begin{enumerate}
	\item We are writing to inform you that we will not be able to process your paper further. Papers sent for peer-review are selected on the basis of discipline, novelty and general significance, in addition to the usual criteria for publication in scholarly journals. Therefore, our decision is not necessarily a reflection of the quality of your research. We wish you every success if you choose to submit the paper elsewhere.
	\item The manuscript should have a structured abstract (Background/ introduction, Methods, Results and Conclusions).
	\item As part of our revised review processes, new submissions can be reviewed by a senior member of the editorial staff for a `fit/no fit' decision.  This can save great time for the authors and avoid lengthy review procedures.  A review of this manuscript has been completed and we do not believe it is a good fit for DSS or its readership.  I see no research contribution in the submission.
	\item I see no research contribution in the submission.  It is a straightforward analysis of one year of a very limited data set.
\end{enumerate}

\section{artikel Ditolak Q3}
Berikut adalah komentar reviewer pada artikel yang ditolak di Q3:
\begin{enumerate}
	\item There is no new idea in the proposed system. The English very poor.The authors should state the contribution of the paper internationally. In addition, requires some native speaker to fix the writing.
	\item Paper has been written on `National Border Agency Communication Behaviour Clustering Using Centrality and Meanshift'. 

	In this paper authors are analysis of boarder security and communication between other countries.What is a contribution in this paper. Not given methodology and algorithm of system. Need to more explanation of results.
	\item The text reports to the problem that often occurs in the Riau Island Province, such as border issues, illegal fishing, drug smuggling, potential transit routes of international terrorism, hazardous waste disposal and human trade, as well as underlying social issues such as health, education, housing and implementation of Asian economic community. As a solution, the author proposes the treatment of the data to generate information
	*********************
	The author did not pay attention to the technical part of the writing, presenting errors as:
	2.1. Eigenvector Centrality 2.2. Eigenvector Centrality
	Equation 1.1.
	**********************
	Errors in writing: detection [14], the certain would be, detection [14].
	**********************
	The text is very weak, with poor results and consequently a weak conclusion. The author does not compare with other algorithms.
	
\end{enumerate}