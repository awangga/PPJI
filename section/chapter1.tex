\chapter{Standar Perlengkapan}

Dalam pertempuran untuk membuat jurnal ilmiah maka diharapkan memeiliki alat bantu berupa aplikasi. Alat bantu aplikasi tersebut berguna untuk proses mempercepat penulisan jurnal ilmiah. Selain aplikasi juga harus memiliki beberapa akun yang berfungsi untuk memperluas jaringan kolaborasi publikasi ilmiah. Beberapa alat bantu aplikasi dan akun yang wajib dimiliki antara lain :
\begin{enumerate}
\item Grammarly dengan akun premium.
\item akun sharelatex dan latex compiler di komputer.
\item akun researchgate
\item Profile Google Scholar
\item Profile orcid.org
\end{enumerate}
Kemudian yang tidak kalah penting adalah memiliki mentor yang mempunya H-Index diatas 10. Memiliki mentor berfungsi untuk mempercepat proses pematangan diri agar siap produktif membuat jurnal. Lebih bagus lagi mentor dari luar negeri. Diharapkan memiliki minimal 3 mentor dari lintas institusi pendidikan.

\section{Pencarian Topik}
Satu-satunya cara adalah untuk mendapatkan topik publikasi adalah dengan membaca jurnal 5 tahun terakhir terindex scopus minimal sebanyak 15 buah. Carilah jurnal dengan pencarian  kata kunci sesuai dengan topik yang kita inginkan. Kemudian tuangkan dalam slide presentasi dengan satu halaman setiap jurnal terdiri dari judul, masalah, metode, hasil.
