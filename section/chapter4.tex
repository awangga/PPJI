\chapter{Langkah Langkah Penulisan Jurnal}

Jika ada pertanyaan darimana saya akan memulai menulis? Maka disinilah jawabannya. Dalam artikel ini akan dijelaskan bagian yang mana yang harus pertama kita lakukan dan tuliskan. Bagian bagian ini berururtan sehingga anda tinggal mengikuti urutan di bab ini. Selain itu di setiap bagian terdapat penilaian untuk mengukur sedalam apa kualitas tulisan dan seperti apa yang harus dituliskan.

\section{Related Works}

Penilaian Realted Works terlihat di Tabel \ref{table:contoh}.
\begin{table}[ht]
\caption{Tabel Related Works}
%\centering
\begin{tabular}{p{1cm}p{5cm}p{5cm}p{1cm}}
\hline
No&Parameter&Bobot&Nilai\\
\hline
1&Jumlah sitasi jurnal terindex scopus pada paragraf  pertama yang memiliki topik yang sama&Per jurnal 2 untuk paraprase dalam bahasa inggris, 1 untuk paraprase dalam bahasa indonesia&\\
2&Jumlah sitasi jurnal terindex scopus di dalam Narasi pada paragraf  kedua yang memiliki metode yang sama&Per jurnal 2 untuk paraprase dalam bahasa inggris, 1 untuk paraprase dalam bahasa indonesia&\\
3&Narasi dalam paragraf ketiga perbedaan antara yang dilakukan sekarang dibandingkan dengan referensi dari nomor 1 dan 2&maksimal 3 untuk tulisan dalam bahasa inggris, 1 untuk tulisan dalam bahasa indonesia&\\
4&Narasi dalam paragraf ketiga dampak adanya penelitian ini&maksimal 3 untuk tulisan dalam bahasa inggris, 1 untuk tulisan dalam bahasa indonesia&\\
5&Narasi dalam paragraf ketiga  tujuan dari penulisan&maksimal 3 untuk tulisan dalam bahasa inggris, 1 untuk tulisan dalam bahasa indonesia&\\
6&Narasi dalam paragraf ketiga sekup dan topik yang dilakukan pada penelitian ini&maksimal 3 untuk tulisan dalam bahasa inggris, 1 untuk tulisan dalam bahasa indonesia&\\
7&Narasi dalam paragraf ketiga metode yang digunakan pada penelitian  ini&maksimal 3 untuk tulisan dalam bahasa inggris, 1 untuk tulisan dalam bahasa indonesia&\\
8&Poin 1-7 dituliskan dalam related works ke dalam jurnal dalam format latex sesuai dengan standar penulisan latex&maksimal 10 untuk tulisan dalam bahasa inggris, 1 untuk tulisan dalam bahasa indonesia&\\
9&Point 1-7 dituliskan di Tinjauan Pustaka pada laporan dalam format latex sesuai dengan standar penulisan latex&maksimal 15 untuk penulisan dalam bahasa inggris, 1 untuk penulisan dalam bahasa indonesia&\\
\hline
\end{tabular}
\label{table:contoh}
\end{table}
  
