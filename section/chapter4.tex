\chapter{Langkah Langkah Penulisan Jurnal}

Jika ada pertanyaan darimana saya akan memulai menulis? Maka disinilah jawabannya. Dalam artikel ini akan dijelaskan bagian yang mana yang harus pertama kita lakukan dan tuliskan. Bagian bagian ini berururtan sehingga anda tinggal mengikuti urutan di bab ini. Selain itu di setiap bagian terdapat penilaian untuk mengukur sedalam apa kualitas tulisan dan seperti apa yang harus dituliskan.

\section{Related Works}

Penilaian Realted Works terlihat di Tabel \ref{table:contoh}.
\begin{longtable}{|p{.03\textwidth}|p{.40\textwidth}|p{.40\textwidth}|p{.10\textwidth}|}
\hline
No&Parameter&Bobot&Nilai\\
\hline
1&Jumlah sitasi jurnal terindex scopus pada paragraf  pertama yang memiliki topik yang sama&Per jurnal 2 untuk paraprase dalam bahasa inggris, 1 untuk paraprase dalam bahasa indonesia&\\ \hline 
2&Jumlah sitasi jurnal terindex scopus di dalam Narasi pada paragraf  kedua yang memiliki metode yang sama&Per jurnal 2 untuk paraprase dalam bahasa inggris, 1 untuk paraprase dalam bahasa indonesia&\\ \hline 
3&Narasi dalam paragraf ketiga perbedaan antara yang dilakukan sekarang dibandingkan dengan referensi dari nomor 1 dan 2&maksimal 3 untuk tulisan dalam bahasa inggris, 1 untuk tulisan dalam bahasa indonesia&\\ \hline 
4&Narasi dalam paragraf ketiga dampak adanya penelitian ini&maksimal 3 untuk tulisan dalam bahasa inggris, 1 untuk tulisan dalam bahasa indonesia&\\ \hline 
5&Narasi dalam paragraf ketiga  tujuan dari penulisan&maksimal 3 untuk tulisan dalam bahasa inggris, 1 untuk tulisan dalam bahasa indonesia&\\ \hline 
6&Narasi dalam paragraf ketiga sekup dan topik yang dilakukan pada penelitian ini&maksimal 3 untuk tulisan dalam bahasa inggris, 1 untuk tulisan dalam bahasa indonesia&\\ \hline 
7&Narasi dalam paragraf ketiga metode yang digunakan pada penelitian  ini&maksimal 3 untuk tulisan dalam bahasa inggris, 1 untuk tulisan dalam bahasa indonesia&\\ \hline 
8&Poin 1-7 dituliskan dalam related works ke dalam jurnal dalam format latex sesuai dengan standar penulisan latex&maksimal 10 untuk tulisan dalam bahasa inggris, 1 untuk tulisan dalam bahasa indonesia&\\ \hline 
9&Point 1-7 dituliskan di Tinjauan Pustaka pada laporan dalam format latex sesuai dengan standar penulisan latex&maksimal 15 untuk penulisan dalam bahasa inggris, 1 untuk penulisan dalam bahasa indonesia&\\ 
\hline

\multicolumn{4}{c}{\textbf{Grammarly Checking}}\\ \hline

10 &Jumlah Contextual Spelling yang bermasalah&setiap jumlah kesalahan dikali minus 3& \\ \hline

11 &Jumlah Grammar yang bermasalah&setiap jumlah kesalahan dikali minus 3& \\ \hline

12 &Jumlah Punctuation yang bermasalah&setiap jumlah kesalahan dikali minus 3& \\ \hline

13 &Jumlah Sentence Structure yang bermasalah&setiap jumlah kesalahan dikali minus 3& \\ \hline

14 &Jumlah Style yang bermasalah&setiap jumlah kesalahan dikali minus 3& \\ \hline

\multicolumn{3}{c}{\textbf{Penilaian}}\\ \hline

15 &Nilai Sebelum Cek Plagiasi&Total nilai di atas& \\ \hline

16 &Persentasi Bebas Plagiarisme&Presentase Uniqnes pada cek plagiasi& \\ \hline

17 &Nilai Akhir&Presentase Uniqnes dikali jumlah total nilai sebelum cek plagiasi& \\ \hline

\caption{Tabel Related Works}
\label{table:contoh}
\end{longtable}



\section{Introduction}

Penilaian Introduction terlihat di Tabel \ref{table:intro}.
\begin{longtable}{|p{.03\textwidth}|p{.40\textwidth}|p{.40\textwidth}|p{.10\textwidth}|}
\hline
No&Parameter&Bobot&Nilai\\
\hline

1 &Jumlah sitasi jurnal atau berita atau UU atau Buku referensi atau ensiklopedia atau buku dokumentasi pada paragraf  pertama yang memaparkan latar belakang&Per sitasi 2 untuk paraprase dalam bahasa inggris, 1 untuk paraprase dalam bahasa indonesia& \\ \hline

2 &Jumlah sitasi jurnal atau berita atau UU atau Buku referensi atau ensiklopedia atau buku dokumentas pada paragraf  kedua yang memaparkan permasalahan&Per sitasi 2 untuk paraprase dalam bahasa inggris, 1 untuk paraprase dalam bahasa indonesia& \\ \hline

3 &Narasi dalam paragraf ketiga yang berisi proposed solution&maksimal 3 untuk tulisan dalam bahasa inggris, 1 untuk tulisan dalam bahasa indonesia& \\ \hline

4 &Sitasi karya dosen dan mahasiswa yang ada di portal if point 11&Per karya nilai 5 tersitasi& \\ \hline

5 &Setiap informasi ada sitasi&maksimal 3 untuk tulisan dalam bahasa inggris, 1 untuk tulisan dalam bahasa indonesia& \\ \hline

6 &Setiap pernyataan ada sitasi&maksimal 3 untuk tulisan dalam bahasa inggris, 1 untuk tulisan dalam bahasa indonesia& \\ \hline

7 &Latar belakang mengandung nyawa seseorang, atau kerugian negara atau kerugian dunia atau permasalahan dunia&maksimal 6 untuk tulisan dalam bahasa inggris, 1 untuk tulisan dalam bahasa indonesia& \\ \hline

8 &Poin 1-7 dituliskan dalam Introduction ke dalam jurnal dalam format latex sesuai dengan standar penulisan latex&maksimal 10 untuk tulisan dalam bahasa inggris, 1 untuk tulisan dalam bahasa indonesia& \\ \hline

9 &Point 1-7 dituliskan di BAB I pada laporan dalam format latex sesuai dengan standar penulisan latex&maksimal 15 untuk penulisan dalam bahasa inggris, 1 untuk penulisan dalam bahasa indonesia& \\ \hline


\multicolumn{4}{c}{\textbf{Grammarly Checking}}\\ \hline

10 &Jumlah Contextual Spelling yang bermasalah&setiap jumlah kesalahan dikali minus 3& \\ \hline

11 &Jumlah Grammar yang bermasalah&setiap jumlah kesalahan dikali minus 3& \\ \hline

12 &Jumlah Punctuation yang bermasalah&setiap jumlah kesalahan dikali minus 3& \\ \hline

13 &Jumlah Sentence Structure yang bermasalah&setiap jumlah kesalahan dikali minus 3& \\ \hline

14 &Jumlah Style yang bermasalah&setiap jumlah kesalahan dikali minus 3& \\ \hline

\multicolumn{3}{c}{\textbf{Penilaian}}\\ \hline

15 &Nilai Sebelum Cek Plagiasi&Total nilai di atas& \\ \hline

16 &Persentasi Bebas Plagiarisme&Presentase Uniqnes pada cek plagiasi& \\ \hline

17 &Nilai Akhir&Presentase Uniqnes dikali jumlah total nilai sebelum cek plagiasi& \\ \hline

\caption{Tabel Introduction}
\label{table:intro}
\end{longtable}



\section{Methods}

Penilaian Methods terlihat di Tabel \ref{table:method}.
\begin{longtable}{|p{.03\textwidth}|p{.40\textwidth}|p{.40\textwidth}|p{.10\textwidth}|}
\hline
No&Parameter&Bobot&Nilai\\
\hline

1 &definisi secara naratif, jelas, dan gamblang dari metode dengan sitasi dari sumber di related works&5 untuk paraprase dalam bahasa inggris, 1 untuk paraprase dalam bahasa indonesia& \\ \hline

2 &algoritma/langkah-langkah secara naratif dalam paragraph atau bentuk algorithma, jelas dan gamblang dari metode dengan sitasi dari sumber yang ada di related works&5 untuk paraprase dalam bahasa inggris, 1 untuk paraprase dalam bahasa indonesia& \\ \hline

3 &penjelasan rumus secara naratif, jelas, dan gamblang dari metode dengan sitasi dari sumber yang ada di related works&5 untuk paraprase dalam bahasa inggris, 1 untuk paraprase dalam bahasa indonesia& \\ \hline

4 &penjelasan naratif cara penerapan metode pada kasus penelitian dengan sitasi dari sumber yang ada di related works&5 untuk paraprase dalam bahasa inggris, 1 untuk paraprase dalam bahasa indonesia& \\ \hline

5 &Menyertakan ilustrasi gambar buatan sendiri dengan resolusi yang tinggi dan bagus. Jika ada tulisan tidak boleh di bawah 9pt. Caption gambar singkat dan jelas&10 untuk gambar dalam bahasa inggris, 1 untuk gambar dalam bahasa indonesia& \\ \hline

6 &Terdapat rumus yang  menggunakan Mathematical expressions format latex dengan tag equation&maksimal 10 untuk satu tag equation& \\ \hline

7 &Pada narasi paragraph penjelasan rumus menggunakan inline penanda rumus \$rumus\$&2 per inline \$ dalam bahasa inggris, 1 dalam bahasa indonesia& \\ \hline

8 &Poin 1-7 dituliskan dalam Method ke dalam jurnal dalam format latex sesuai dengan standar penulisan latex&maksimal 10 untuk tulisan dalam bahasa inggris, 1 untuk tulisan dalam bahasa indonesia& \\ \hline

9 &Point 1-7 dituliskan di BAB III pada laporan dalam format latex sesuai dengan standar penulisan latex&maksimal 15 untuk penulisan dalam bahasa inggris, 1 untuk penulisan dalam bahasa indonesia& \\ \hline


\multicolumn{4}{c}{\textbf{Grammarly Checking}}\\ \hline

10 &Jumlah Contextual Spelling yang bermasalah&setiap jumlah kesalahan dikali minus 3& \\ \hline

11 &Jumlah Grammar yang bermasalah&setiap jumlah kesalahan dikali minus 3& \\ \hline

12 &Jumlah Punctuation yang bermasalah&setiap jumlah kesalahan dikali minus 3& \\ \hline

13 &Jumlah Sentence Structure yang bermasalah&setiap jumlah kesalahan dikali minus 3& \\ \hline

14 &Jumlah Style yang bermasalah&setiap jumlah kesalahan dikali minus 3& \\ \hline

\multicolumn{3}{c}{\textbf{Penilaian}}\\ \hline

15 &Nilai Sebelum Cek Plagiasi&Total nilai di atas& \\ \hline

16 &Persentasi Bebas Plagiarisme&Presentase Uniqnes pada cek plagiasi& \\ \hline

17 &Nilai Akhir&Presentase Uniqnes dikali jumlah total nilai sebelum cek plagiasi& \\ \hline

\caption{Tabel Method}
\label{table:method}
\end{longtable}

\section{Experiments}

Penilaian Experiments terlihat di Tabel \ref{table:experiments}.
\begin{longtable}{|p{.03\textwidth}|p{.40\textwidth}|p{.40\textwidth}|p{.10\textwidth}|}
\hline
No&Parameter&Bobot&Nilai\\
\hline

1 &penjelasan secara naratif, jelas, dan gamblang dari sumber data, jumlah data dan kejelasan jenis dan isi data&5 untuk paraprase dalam bahasa inggris, 1 untuk paraprase dalam bahasa indonesia& \\ \hline

2 &penjelasan secara naratif(Input-Proses-Output) bagaimana metode diterapkan dengan data nomor 1 dalam paragraph, jelas dan gamblang&5 untuk paraprase dalam bahasa inggris, 1 untuk paraprase dalam bahasa indonesia& \\ \hline

3 &Jika sudah ada hasil eksperiment bonus 20&20 untuk paraprase dalam bahasa inggris, 1 untuk paraprase dalam bahasa indonesia& \\ \hline

4 &Penjelasan secara naratif lokasi studi kasus yang menjadi eksperimen dikaitkan dengan background&5 untuk paraprase dalam bahasa inggris, 1 untuk paraprase dalam bahasa indonesia& \\ \hline

5 &Menyertakan ilustrasi gambar buatan sendiri atau foto alat eksperimen atau skrinsut aplikasi atau foto ujicoba dengan resolusi yang tinggi dan bagus. Jika ada tulisan tidak boleh di bawah 9pt. Caption gambar singkat dan jelas&10 untuk gambar dalam bahasa inggris, 1 untuk gambar dalam bahasa indonesia& \\ \hline

6 &ekperimen dibantu dengan menggunakan bahasa pemrograman&maksimal 10& \\ \hline

7 &bahasa pemrograman dibuat sendiri untuk pengerjaan eksperimen sesuai dengan metode bukan plagiasi dari tempat lain&maksimal 10& \\ \hline

8 &Poin 1-7 dituliskan dalam Eksperiment ke dalam jurnal dalam format latex sesuai dengan standar penulisan latex&maksimal 10 untuk tulisan dalam bahasa inggris, 1 untuk tulisan dalam bahasa indonesia& \\ \hline

9 &Point 1-7 dituliskan di BAB IV pada laporan dalam format latex sesuai dengan standar penulisan latex&maksimal 15 untuk penulisan dalam bahasa inggris, 1 untuk penulisan dalam bahasa indonesia& \\ \hline


\multicolumn{4}{c}{\textbf{Grammarly Checking}}\\ \hline

10 &Jumlah Contextual Spelling yang bermasalah&setiap jumlah kesalahan dikali minus 3& \\ \hline

11 &Jumlah Grammar yang bermasalah&setiap jumlah kesalahan dikali minus 3& \\ \hline

12 &Jumlah Punctuation yang bermasalah&setiap jumlah kesalahan dikali minus 3& \\ \hline

13 &Jumlah Sentence Structure yang bermasalah&setiap jumlah kesalahan dikali minus 3& \\ \hline

14 &Jumlah Style yang bermasalah&setiap jumlah kesalahan dikali minus 3& \\ \hline

\multicolumn{3}{c}{\textbf{Penilaian}}\\ \hline

15 &Nilai Sebelum Cek Plagiasi&Total nilai di atas& \\ \hline

16 &Persentasi Bebas Plagiarisme&Presentase Uniqnes pada cek plagiasi& \\ \hline

17 &Nilai Akhir&Presentase Uniqnes dikali jumlah total nilai sebelum cek plagiasi& \\ \hline

\caption{Tabel Experiments}
\label{table:experiments}
\end{longtable}



\section{Results}

Penilaian Results terlihat di Tabel \ref{table:results}.
\begin{longtable}{|p{.03\textwidth}|p{.40\textwidth}|p{.40\textwidth}|p{.10\textwidth}|}
\hline
No&Parameter&Bobot&Nilai\\
\hline

1 &The results section always begins with text, reporting the key results and referring to your figures and tables as you proceed&10 untuk kalimat dalam bahasa inggris, 1 untuk kalimat dalam bahasa indonesia& \\ \hline

2 &Present the results of your experiment(s) in a sequence that will logically support (or provide evidence against) the hypothesis, or answer the question, stated in the Introduction.&10 untuk kalimat dalam bahasa inggris, 1 untuk kalimat dalam bahasa indonesia& \\ \hline

3 &The body of the Results section is a text-based presentation of the key findings which includes references to each of the Tables and Figures. The text should guide the reader through your results stressing the key results which provide the answers to the question(s) investigated. A major function of the text is to provide clarifying information. You must refer to each Table and/or Figure individually and in sequence (using latex ref), and clearly indicate for the reader the key results that each conveys. Key results depend on your questions, they might include obvious trends, important differences, similarities, correlations, maximums, minimums, etc.&10 untuk kalimat dalam bahasa inggris, 1 untuk kalimat dalam bahasa indonesia& \\ \hline

4 &Summaries of the statistical analyses may appear either in the text (usually parenthetically) or in the relevant Tables or Figures (in the legend or as footnotes to the Table or Figure) &10 untuk kalimat dalam bahasa inggris, 1 untuk kalimat dalam bahasa indonesia& \\ \hline

5 &Cek semua tulisan termasuk tulisan sebelumnya : Satu paragraph minimal 3 kalimat, setiap gambar/tabel harus direferensikan di kalimat.tidak boleh melakukan referensi relatif &10 untuk kalimat dalam bahasa inggris, 1 untuk kalimat dalam bahasa indonesia &  \\ \hline

6 &Use and over-use of the word \"significant\": Your results will read much more cleanly if you avoid overuse of the word siginifcant in any of its forms.&10 untuk kalimat dalam bahasa inggris, 1 untuk kalimat dalam bahasa indonesia& \\ \hline

7 &Always enter the appropriate units when reporting data or summary statistics.for an individual value you would write, \"the mean length was 10 m\", or, "the maximum time was 140 min. & 10 untuk kalimat dalam bahasa inggris, 1 untuk kalimat dalam bahasa indonesia& \\ \hline

8 &Cek semua tulisan termasuk tulisan sebelumnya : Satu paragraph minimal 3 kalimat, setiap gambar/tabel harus direferensikan di kalimat.tidak boleh melakukan referensi relatif & 5 jika sudah benar seluruh tulisan di jurnal & \\ \hline

9 &Poin 1-7 dituliskan dalam Results ke dalam jurnal dalam format latex sesuai dengan standar penulisan latex & maksimal 10 untuk tulisan dalam bahasa inggris, 1 untuk tulisan dalam bahasa indonesia & \\ \hline

10 &Point 1-7 dituliskan di BAB V pada laporan dalam format latex sesuai dengan standar penulisan latex & maksimal 15 untuk penulisan dalam bahasa inggris, 1 untuk penulisan dalam bahasa indonesia & \\ \hline


\multicolumn{4}{c}{\textbf{Grammarly Checking}}\\ \hline

11 &Jumlah Contextual Spelling yang bermasalah&setiap jumlah kesalahan dikali minus 3& \\ \hline

12 &Jumlah Grammar yang bermasalah&setiap jumlah kesalahan dikali minus 3& \\ \hline

13 &Jumlah Punctuation yang bermasalah&setiap jumlah kesalahan dikali minus 3& \\ \hline

14 &Jumlah Sentence Structure yang bermasalah&setiap jumlah kesalahan dikali minus 3& \\ \hline

15 &Jumlah Style yang bermasalah&setiap jumlah kesalahan dikali minus 3& \\ \hline

\multicolumn{3}{c}{\textbf{Penilaian}}\\ \hline

16 &Nilai Sebelum Cek Plagiasi&Total nilai di atas& \\ \hline

17 &Persentasi Bebas Plagiarisme&Presentase Uniqnes pada cek plagiasi& \\ \hline

18 &Nilai Akhir&Presentase Uniqnes dikali jumlah total nilai sebelum cek plagiasi& \\ \hline

\caption{Tabel Results}
\label{table:results}
\end{longtable}




\section{Conclussion}

Penilaian Conclussion terlihat di Tabel \ref{table:conclussion}.
\begin{longtable}{|p{.03\textwidth}|p{.40\textwidth}|p{.40\textwidth}|p{.10\textwidth}|}
\hline
No&Parameter&Bobot&Nilai\\
\hline

1 &The research questions have been answered. & 6 untuk kalimat dalam bahasa inggris, 1 untuk kalimat dalam bahasa indonesia & \\ \hline

2 &The main question or problem statement has been answered. & 6 untuk kalimat dalam bahasa inggris, 1 untuk kalimat dalam bahasa indonesia & \\ \hline

3 &The hypotheses have been confirmed or refused. & 6 untuk kalimat dalam bahasa inggris, 1 untuk kalimat dalam bahasa indonesia & \\ \hline

4 &The right verb tense has been used. & 6 untuk kalimat dalam bahasa inggris, 1 untuk kalimat dalam bahasa indonesia & \\ \hline

5 &No issues are interpreted. & 6 untuk kalimat dalam bahasa inggris, 1 untuk kalimat dalam bahasa indonesia & \\ \hline

6 &No new information has been given. & 6 untuk kalimat dalam bahasa inggris, 1 untuk kalimat dalam bahasa indonesia & \\ \hline

7 &No examples are used.& 6 untuk kalimat dalam bahasa inggris, 1 untuk kalimat dalam bahasa indonesia & \\ \hline

8 &No extraneous information is provided. & 6 untuk kalimat dalam bahasa inggris, 1 untuk kalimat dalam bahasa indonesia & \\ \hline

9 &No passages from the results have been cut and pasted. & 6 untuk kalimat dalam bahasa inggris, 1 untuk kalimat dalam bahasa indonesia & \\ \hline

10 &Mereferensikan gambar atau tabel dari result & 6 untuk kalimat dalam bahasa inggris, 1 untuk kalimat dalam bahasa indonesia & \\ \hline

11 &present comparison between porformance of your approach with related works(With CItation) & 6 untuk kalimat dalam bahasa inggris, 1 untuk kalimat dalam bahasa indonesia & \\ \hline

12 &Prove that your manuscript has a significant value not trivial & 6 untuk kalimat dalam bahasa inggris, 1 untuk kalimat dalam bahasa indonesia & \\ \hline

13 &Poin 1-7 dituliskan dalam Conclusion ke dalam jurnal dalam format latex sesuai dengan standar penulisan latex & maksimal 13 untuk tulisan dalam bahasa inggris, 1 untuk tulisan dalam bahasa indonesia & \\ \hline

14 &Point 1-7 dituliskan di BAB VI pada laporan dalam format latex sesuai dengan standar penulisan latex & maksimal 15 untuk penulisan dalam bahasa inggris, 1 untuk penulisan dalam bahasa indonesia & \\ \hline


\multicolumn{4}{c}{\textbf{Grammarly Checking}}\\ \hline

15 &Jumlah Contextual Spelling yang bermasalah&setiap jumlah kesalahan dikali minus 3& \\ \hline

16 &Jumlah Grammar yang bermasalah&setiap jumlah kesalahan dikali minus 3& \\ \hline

17 &Jumlah Punctuation yang bermasalah&setiap jumlah kesalahan dikali minus 3& \\ \hline

18 &Jumlah Sentence Structure yang bermasalah&setiap jumlah kesalahan dikali minus 3& \\ \hline

19 &Jumlah Style yang bermasalah&setiap jumlah kesalahan dikali minus 3& \\ \hline

\multicolumn{3}{c}{\textbf{Penilaian}}\\ \hline

20 &Nilai Sebelum Cek Plagiasi&Total nilai di atas& \\ \hline

21 &Persentasi Bebas Plagiarisme&Presentase Uniqnes pada cek plagiasi& \\ \hline

22 &Nilai Akhir&Presentase Uniqnes dikali jumlah total nilai sebelum cek plagiasi& \\ \hline

\caption{Tabel Conclussion}
\label{table:conclussion}
\end{longtable}


\section{Discussion}

Penilaian Discussion terlihat di Tabel \ref{table:Discussion}.
\begin{longtable}{|p{.03\textwidth}|p{.40\textwidth}|p{.40\textwidth}|p{.10\textwidth}|}
\hline
No&Parameter&Bobot&Nilai\\
\hline

1 &The validity of the research is demonstrated. & 6 untuk kalimat dalam bahasa inggris, 1 untuk kalimat dalam bahasa indonesia& \\ \hline

2 &New insights are explained. & 6 untuk kalimat dalam bahasa inggris, 1 untuk kalimat dalam bahasa indonesia & \\ \hline

3 &The limitations of the research are discussed.& 6 untuk kalimat dalam bahasa inggris, 1 untuk kalimat dalam bahasa indonesia & \\ \hline

4 &It is indicated whether expectations were justified. & 6 untuk kalimat dalam bahasa inggris, 1 untuk kalimat dalam bahasa indonesia & \\ \hline

5 &Possible causes and consequences of the results are discussed. & 6 untuk kalimat dalam bahasa inggris, 1 untuk kalimat dalam bahasa indonesia & \\ \hline

6 &Suggestions for possible follow-up research are made. & 6 untuk kalimat dalam bahasa inggris, 1 untuk kalimat dalam bahasa indonesia & \\ \hline

7 &Own interpretations have been included in the discussion. & 6 untuk kalimat dalam bahasa inggris, 1 untuk kalimat dalam bahasa indonesia & \\ \hline

8 &There are no suggestions for follow-up research that are too vague. & 6 untuk kalimat dalam bahasa inggris, 1 untuk kalimat dalam bahasa indonesia & \\ \hline

9 &Cek tulisan seluruhnya di jurnal : pastikan tidak ada kata ganti orang pertama kedua, ketiga. Contoh : the researcher, the author, I, they, you, we, us etc. & 6 untuk kalimat dalam bahasa inggris, 1 untuk kalimat dalam bahasa indonesia & \\ \hline

10 &Cek tulisan seluruhnya di jurnal : setiap gambar atau tabel pastikan di referensikan dalam kalimat di dalam paragraf yang utuh.& 6 untuk kalimat dalam bahasa inggris, 1 untuk kalimat dalam bahasa indonesia & \\ \hline

11 &Cek tulisan seluruhnya di jurnal : pastikan satu paragraf terdiri minimal 3 kalimat. & 6 untuk kalimat dalam bahasa inggris, 1 untuk kalimat dalam bahasa indonesia& \\ \hline

12 &Cek tulisan seluruhnya di jurnal : pastikan tidak ada penunjukan relatif. Contoh : tabel di bawah ini, gambar di atas, gambar di bawah ini. & 6 untuk kalimat dalam bahasa inggris, 1 untuk kalimat dalam bahasa indonesia&  \\ \hline

13 &Poin 1-7 dituliskan dalam Discussion ke dalam jurnal dalam format latex sesuai dengan standar penulisan latex & maksimal 13 untuk tulisan dalam bahasa inggris, 1 untuk tulisan dalam bahasa indonesia & \\ \hline

14 &Point 1-7 dituliskan di BAB VII pada laporan dalam format latex sesuai dengan standar penulisan latex & maksimal 15 untuk penulisan dalam bahasa inggris, 1 untuk penulisan dalam bahasa indonesia & \\ \hline


\multicolumn{4}{c}{\textbf{Grammarly Checking}}\\ \hline

15 &Jumlah Contextual Spelling yang bermasalah&setiap jumlah kesalahan dikali minus 3& \\ \hline

16 &Jumlah Grammar yang bermasalah&setiap jumlah kesalahan dikali minus 3& \\ \hline

17 &Jumlah Punctuation yang bermasalah&setiap jumlah kesalahan dikali minus 3& \\ \hline

18 &Jumlah Sentence Structure yang bermasalah&setiap jumlah kesalahan dikali minus 3& \\ \hline

19 &Jumlah Style yang bermasalah&setiap jumlah kesalahan dikali minus 3& \\ \hline

\multicolumn{3}{c}{\textbf{Penilaian}}\\ \hline

20 &Nilai Sebelum Cek Plagiasi&Total nilai di atas& \\ \hline

21 &Persentasi Bebas Plagiarisme&Presentase Uniqnes pada cek plagiasi& \\ \hline

22 &Nilai Akhir&Presentase Uniqnes dikali jumlah total nilai sebelum cek plagiasi& \\ \hline

\caption{Tabel Discussion}
\label{table:Discussion}
\end{longtable}

\section{Abstract}

Penilaian Abstract terlihat di Tabel \ref{table:Abstract}.
\begin{longtable}{|p{.03\textwidth}|p{.40\textwidth}|p{.40\textwidth}|p{.10\textwidth}|}
\hline
No&Parameter&Bobot&Nilai\\
\hline


1 &terdiri dari 150-200 kata tanpa ada sitasi&6 untuk kalimat dalam bahasa inggris, 1 untuk kalimat dalam bahasa indonesia& \\ \hline

2 &berisi latar belakang yang penting dan jelas&6 untuk kalimat dalam bahasa inggris, 1 untuk kalimat dalam bahasa indonesia& \\ \hline

3 &Berisi tujuan yang singkat padas jelas&6 untuk kalimat dalam bahasa inggris, 1 untuk kalimat dalam bahasa indonesia& \\ \hline

4 &berisi metode yang diusulkan dan digunakan&6 untuk kalimat dalam bahasa inggris, 1 untuk kalimat dalam bahasa indonesia& \\ \hline

5 &jelaskan secara singkat dan jelas hasil yang didapatkan&6 untuk kalimat dalam bahasa inggris, 1 untuk kalimat dalam bahasa indonesia& \\ \hline

6 &jelaskan secara singkat dan jelas kesimpulan dari penelitian&6 untuk kalimat dalam bahasa inggris, 1 untuk kalimat dalam bahasa indonesia& \\ \hline

7 &jelaskan secara singkat dan jelas saran untuk pengembangan penelitian selanjutnya&6 untuk kalimat dalam bahasa inggris, 1 untuk kalimat dalam bahasa indonesia& \\ \hline

8 &alur mudah dimengerti dan dicerna serta menggunakan kalimat yang teratur dan terstruktur.&6 untuk kalimat dalam bahasa inggris, 1 untuk kalimat dalam bahasa indonesia& \\ \hline

9 &harus dimunculkan persoalan utama dan pentingnya melakukan penelitian ini, serta solusi yang diusulkan&6 untuk kalimat dalam bahasa inggris, 1 untuk kalimat dalam bahasa indonesia& \\ \hline

10 &isi tertuang dengan kalimat yangjelas&6 untuk kalimat dalam bahasa inggris, 1 untuk kalimat dalam bahasa indonesia& \\ \hline

11 &kata kunci atau keyword ditentukan denagn nama metode yang digunakan dan sub sub bidang penelitian yang dilakukan&6 untuk kalimat dalam bahasa inggris, 1 untuk kalimat dalam bahasa indonesia& \\ \hline

12 &kata kunci minimal 5, berkaitan dengan metode dan sub sub sub bidang penelitian yang dilakukan&6 untuk kalimat dalam bahasa inggris, 1 untuk kalimat dalam bahasa indonesia& \\ \hline

13 &Poin 1-7 dituliskan dalam Abstract ke dalam jurnal dalam format latex sesuai dengan standar penulisan latex&maksimal 13 untuk tulisan dalam bahasa inggris, 1 untuk tulisan dalam bahasa indonesia& \\ \hline

14 &Point 1-7 dituliskan di Abstract pada laporan dalam format latex sesuai dengan standar penulisan latex&maksimal 15 untuk penulisan dalam bahasa inggris, 1 untuk penulisan dalam bahasa indonesia& \\ \hline


\multicolumn{4}{c}{\textbf{Grammarly Checking}}\\ \hline

15 &Jumlah Contextual Spelling yang bermasalah&setiap jumlah kesalahan dikali minus 3& \\ \hline

16 &Jumlah Grammar yang bermasalah&setiap jumlah kesalahan dikali minus 3& \\ \hline

17 &Jumlah Punctuation yang bermasalah&setiap jumlah kesalahan dikali minus 3& \\ \hline

18 &Jumlah Sentence Structure yang bermasalah&setiap jumlah kesalahan dikali minus 3& \\ \hline

19 &Jumlah Style yang bermasalah&setiap jumlah kesalahan dikali minus 3& \\ \hline

\multicolumn{3}{c}{\textbf{Penilaian}}\\ \hline

20 &Nilai Sebelum Cek Plagiasi&Total nilai di atas& \\ \hline

21 &Persentasi Bebas Plagiarisme&Presentase Uniqnes pada cek plagiasi& \\ \hline

22 &Nilai Akhir&Presentase Uniqnes dikali jumlah total nilai sebelum cek plagiasi& \\ \hline

\caption{Tabel Abstract}
\label{table:Abstract}
\end{longtable}
