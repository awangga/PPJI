\chapter{Standar Penulisan Jurnal}

Penulisan jurnal harus menggunakan Latex dengan template IEEE. Template bisa di unduh pada menu dokumen portal kampus keren atau situs informatika. Penulisan jurnal harus mengikuti standar penulisan akademis dan mengikuti kerangka jurnal. Jurnal wajib menggunakan bahasa inggris(Amerika) yang dikoreksi bersama pembimbing atau kolaborator. 

\section{Standar Penulisan}
Di dalam penulisan artikel ilmiah harus mengikuti standar minimal penulisan ilmiah. Standar ini digunakan untuk menyamakan semantik bahasa agar tulisan lebih mudah dibaca dan dipahami. Penggunaan standar merupakan keniscayaan dalam penulisan artikel ilmiah.
\subsection{Penggunaan Kalimat}
Penulisan jurnal harus menggunakan kalimat aktif dan positif. Memiliki Subject, Predikat dan Object yang jelas. Tidak bertele-tele dan terlalu panjang dalam penggunaan kalimat(terlalu banyak kata sambung dan tanda koma). Satu paragraf minimal terdiri dari tiga kalimat. Hindari paragraph yang terdiri dari satu kalimat yang biasanya digunakan untuk penjelasan gambar, rumus atau tabel. Lebih baik digabungkan saja dengan narasi paragraph sebelumnya. Jika memang harus ada penjelasan kalimat, maka kembangkan lagi menjadi narasi satu paragraph utuh. Tidak boleh menulis kata ganti orang seperi Penulis, Saya, Kami, Mereka. Gunakan kata benda seperti penelitian ini, riset ini.

\subsection{Penempatan Sitasi}
Sitasi ditempatkan tepat pada akhir kalimat penjelasan referensi sebelum tanda 
pemisah antar kalimat (koma atau titik) tanpa spasi. 
Sitasi juta dapat ditumpuk pada sebuah kalimat yang merupaka  penjelasan singkat dari referensi. 
Please ensure that: all references have been cited in your text. Each citation should be written in the order of appearance in the text. The references must be presented in numbering. 

\subsection{gambar, rumus, tabel}
Pemanfaatan instrumen pendukung gambar kualitasnya harus ditingkatkan, jangan sampai terdapat gambar yang tidak bisa terbaca tulisannya.
Tidak diperbolehkan memberikan narasi penunjukan relatif. seperti :
\begin{itemize}
	\item Lebih detailnya lihat gambar di bawah ini
	\item Untuk lebih jelasnya lihar rumus di bawah ini
	\item data bisa dilihat di tabel di atas
\end{itemize}
Diperbaiki yang seharusnya :
\begin{itemize}
	\item Pada gambar 1.1 terlihat bahwa hasil perhitungan penduduk sudah mulai jenuh.
	\item Total kejenuhan hasil kalkulasi terlihat di tabel 1.1.
	\item Rumus 1.1 merupakan rumus kalkulasi tingkat kejenuhan.
\end{itemize}
Prepare your figures in high quality and created by yourself (not copy and paste from other parties). All legends, captions, etc in your figures MUST in English.


\section{Kerangka Jurnal}
Kerangka acuan dalam membuat jurnal harus memenuhi standar acuan di sub bab ini. Masing-masing kerangka jurnal harus memenuhi standar dan aturan yang ditetapkan. Pengerjaan jurnal biasanya lebih awal daripada pengerjaan laporan.Bagian-bagian dari jurnal terdiri dari abstrak (Abstract), pendahuluan (Introduction), metode (Methods), Penelitian Terkait (Related Works),percobaan (Experiment), hasil (Result) dan diskusi (Discussion).

\subsection{Judul}
Maksimal 10 (sepuluh) kata dalam Bahasa Inggris ringkas dan tegas.

\subsection{Abstract}
Terdiri dari 150-200 kata tanpa ada sitasi. Berisi latar belakang, tujuan,metode, hasil,kesimpulan dan saran. Pada abstrak harus dimunculkan persoalan utama dan pentingnya melakukan penelitian ini, serta solusi yang diusulkan. Isi tertuang dengan kalimat yang jelas.Kata kunci atau keyword ditentukan dengan nama metode yang digunakan dan sub sub bidang penelitian yang dilakukan. Kata kunci minimal harus terdapat lima kata kunci.

\subsection{Introduction}
Pada bagian pendahuluan uraikan rincian persoalan terkini berdasarkan beberapa referensi dari jurnal intenasional yang di sitasi, sehingga penelitian ini layak dilakukan. An Introduction should contain the following three parts:
\begin{itemize}
    \item Background: Authors have to make clear what the context is. Ideally, authors should give an idea of the state-of-the art of the field the report is about.
    \item  The Problem: If there was no problem, there would be no reason for writing a manuscript, and definitely no reason for reading it. So, please tell readers why they should proceed reading. Experience shows that for this part a few lines are often sufficient.
    \item The Proposed Solution: Now and only now! - authors may outline the contribution of the manuscript. Here authors have to make sure readers point out what are the novel aspects of authors work.
\end{itemize}
Authors should place the paper in proper context by citing relevant papers. Setiap ada pemaparan data, informasi, dan sebuah pernyataan pada sebuah kalimat maka wajib diakhiri dengan sitasi.Minimal terdapat sitasi pada setiap kalimat pernyataan, informasi, dan data pada bagian Introduction dari jurnal 5 tahun terakhir terindex scopus.

\subsection{Related Works}
Penjelasan singkat dengan sitasi dari artikel yang direferensikan minimal dari 10 artikel. Artikel yang dijelaskan merupakan artikel yang terkait dengan kata kunci penelitian. Minimal 3 Paragraph. Pada paragraph terakhir harus ada pernyataan perbedaan antara penelitian yang akan dilakukan dengan penelitian yang disebutkan pada sitasi di related works.

\subsection{Method}
Penjelasan teknis yang jelas dan gamblang mengenai metode yang digunakan dengan sitasi. Terdiri dari definisi, konsep, rumus atau diagram. Metode yang digunakan adalah metode yang terdapat pada referensi dalam 5 tahun terakhir  dari  jurnal  internasional  terindex  scopus. 

\subsection{Experiment}
Data sumber yang jelas dan cukup untuk dijadikan penelitian, disertai dengan hasil nya sesuai langkah-langkah yang di tuliskan di Method.

\subsection{Result and Discussion}
Sangat jelas relevasinya dengan latar belakang dan pembahasan, dirumuskan dengan singkat. The presentation of results should be simple and straightforward in style. You should improve your analyzing and also present the comparison between performance of your approach and other researches. Results given in figures should not be repeated in tables. This section report the most important findings, including results of analyses as appropriate. It is very important to prove that your manuscript has a significant value and not trivial.

\subsection{Reference}
Semua  referensi  yang  digunakan  harus  terindex  pada  google  scholar.   Minimal  15 referensi dari jurnal terindeks Scopus dan merupakan artikel dalam 5 tahun terakhir. Jurnal yang terindex scopus bisa dicek di situs scimagojr.com.  Format referensi yang disikan  pada  formulir  pengajuan  penelitian  tingkat  akhir  adalah  BibTex.   BibTex referensi bisa didapatkan pada laman Google Scholar.

\section{Standar Format Latex}
Beberapa Aturan yang harus dipatuhi :
\begin{enumerate}

    \item file disimpan dalam format ber ekstensi .tex per chapter masing2 di folder section

    \item gambar disimpan dalam folder figures dengan namagambar

    \item referensi dari google scholar,scholar.google.com

    \item Setiap referensi yang diambil, maka tambahkan dan tuliskan ke dalam file bernama references.bib yang berisi kumpulan bibTex dari referensi. Gunakan standar pengutipan yang baik dan benar

    \item Gambar disebutkan di dalam artikel dengan format sesuai labelnya yaitu \\ \verb|\ref{labelgambar}|. \\ Gambar diselipkan dengan menambahkan blok sintaks :
    \begin{verbatim}
    \begin{figure}[ht]
    \centerline{\includegraphics[width=1\textwidth]
    {figures/namagambar.JPG}}
    \caption{penjelasan keterangan gambar.}
    \label{labelgambar}
    \end{figure}
    
    Contoh :
    Pada gambar \ref{labelgambar} dijelaskan bahwa 
    sistem operasi memiliki 3 versi.
    \end{verbatim}

    \item Referensi disebutkan dengan menyebutkan nama di dalam file bibtex No.4. \\
    Contoh, Jika Bibtex sudah diinputkan kedalam reference.bib seperti ini :
    \begin{verbatim}
    @inproceedings{ganapathi2006windows,
      title={Windows XP Kernel Crash Analysis.},
      author={Ganapathi, Archana and Ganapathi, 
      Viji and Patterson, David A},
      booktitle={LISA},
      volume={6},
      pages={49--159},
      year={2006}
    }
    \end{verbatim}
    Maka penulisan kalimat di jurnal : \\
    Dalam sebuah artikel dari Ganapathi yang 
    menyebutkan bahwa komputasi adalah keniscayan \verb|\cite{ganapathi2006windows}|.
    
    \item Penyebutan subbab dan subsubbab diatur dengan cara : \\
    judul sub bab : \\ 
    \verb|\section{nama sub bab}| \\
    judul sub sub bab ditulis dengan :\\ 
    \verb|\subsection{judul sub sub bab} | \\
    judul sub sub sub bab ditulis dengan : \\ \verb|\subsubsection{Judul sub sub sub bab} | \\
    contoh :
    \begin{verbatim}
    \section{Sejarah Peta}
Perkembangan peta dunia tidak luput dari para ahli 
geografi dan kartografi. Peta dunia yang populer pada saat 
ini merupakan 
kontribusi dari para 
pembuat peta sebelumnya

\subsection{Ptolemy's}
Ptolemy's diduga membuat peta pada abad ke 2
\end{verbatim}
    
    \item untuk list dan nomor gunakan enumerate atau itemize contoh :
    \begin{verbatim}
berikut nama anggota kelompok
\begin{enumerate}
\item darso
\item karyo
\item doyok
\end{enumerate}

\begin{enumerate}
\item
This is the first item in the numbered list.

\item
This is the second item in the numbered list.
\end{enumerate}

\begin{itemize}
\item
This is the first item in the itemized list.

\item
This is the first item in the itemized list.
This is the first item in the itemized list.
This is the first item in the itemized list.
\end{itemize}

\begin{itemize}
\item[]
This is the first item in the itemized list.

\item[]
This is the first item in the itemized list.
This is the first item in the itemized list.
This is the first item in the itemized list.
\end{itemize}
    \end{verbatim}
    
    \item spesial karakter menggunakan tanda `\verb|\|' didepannya contoh :
    \begin{verbatim}
\& 
\% 
\$ 
\#  
\{ \}
\_
\"dalam petik\"
`dalam petik'
jika spesial karakter menjadi banyak atau satu baris gunakan verb
contoh :
\verb|%$'%&$&'%'%'%&'%|
    \end{verbatim}
    
    \item untuk tabel gunakan table , dan jangan lupa tabel di referensikan pada kalimat berdasarkan labelnya. contoh:
    \begin{verbatim}
ini merupakan contoh tabel \ref{table:contoh} ukuran kecil.
\begin{table}[h]
\caption{Small Table}
\centering
\begin{tabular}{ccc}
\hline
one&two&three\\
\hline
C&D&E\\
\hline
\end{tabular}
\label{table:contoh}
\end{table}
    \end{verbatim}
    
    \item untuk rumus gunakan tag equation dan di referensikan pada kalimat dengan tag ref sesuai labelnya contoh:
    \begin{verbatim}
Luas permukaan dijelaskan pada rumus \ref{eq:1}.Volume dijelaskan 
pada rumus \ref{eq:2}.
$L$ merupakan luas, $\pi$ adalah 3,14.
\begin{equation}\label{eq:1}
     L = 4 \pi r^2 \,
\end{equation}
 \begin{equation}\label{eq:2}
     V = \frac{4}{3}\pi r^3
\end{equation}
    \end{verbatim}
    
    \item untuk kode program menggunakan verbatim
    \begin{verbatim}
\ begin{verbatim}
a = "anu"
b = "itu"
c = a + b
print(c) 
\ end{verbatim}
    \end{verbatim}
\end{enumerate}




